\documentclass{article}
\usepackage{fontspec}
\usepackage{amsmath, amsthm, amssymb}
\usepackage[top=2cm,bottom=2cm,left=2.5cm,right=2cm]{geometry}

\usepackage{polyglossia}
\setdefaultlanguage{hebrew}
\setotherlanguages{english}

\newfontfamily\hebrewfont{David}

\setlength{\parindent}{0em}
\setlength{\parskip}{0.8em}

\newtheorem{theorem}{משפט}
\newtheorem{lemma}{למה}
\newtheorem{corollary}{מסקנה}
\newtheorem*{theorem*}{משפט}

\theoremstyle{definition}
\newtheorem*{definition*}{הגדרה}
\newtheorem*{example*}{דוגמה}

\newcommand{\ncmd}{\newcommand}

%\ncmd{\todo}[1]{}
\ncmd{\todo}[1]{\textbf{TODO #1}}

\ncmd{\prt}{\partial}

\ncmd{\mbb}[1]{\mathbb{#1}}
\ncmd{\R}{\mbb{R}}
\ncmd{\mrm}[1]{\mathrm{#1}}
\ncmd{\dd}[1]{\mrm{d}#1}

\ncmd{\vphi}{\varphi}

\DeclareMathOperator{\im}{Im}
\DeclareMathOperator{\spn}{span}

\ncmd{\inj}{\hookrightarrow}

\newcommand{\norm}[1]{\left\lVert#1\right\rVert}

\title{סמינר מורס – קיומן של פונקציות מורס}
\author{שי בן משה}
\date{21/11/2016}

\begin{document}
	\maketitle
	
	
	
	
	\section{הקדמה}
	
	מטרת ההרצאה הזו היא להוכיח את קיומן של פונקציות מורס על יריעה חלקה \(M\) ממימד \(k\).
	
	לצורך כך, נזכר במשפט יסודי מהתורה של יריעות חלקות:
	\begin{theorem*}[משפט השיכון של וויטני]
		קיים \(n\) כך שקיים שיכון חלק
		\(M \inj \R^n\).
	\end{theorem*}
	נניח מעתה כי 
	\(M \subset \R^n\)
	והעתקות
	\(x=(x^1,\dotsc,x^n): \R^n \to \R^n\)
	יסמנו את ההטלות לקורדינטות.
	
	\begin{definition*}
		עבור
		\(p\in \R^n\)
		נגדיר
		\(L_p:M\to \R\)
		על ידי
		\(L_p(q)=\norm{p-q}^2\).
	\end{definition*}

	נוכיח את המשפט הבא:
	\begin{theorem}
		לכל
		\(p\in \R^n\),
		פרט לקבוצה ממידה 0, \(L_p\) היא פונקציית מורס.
	\end{theorem}




	\section{מרחב נורמלי}
	
	כדי להגדיר דברים בקורדינטות, נניח שיש לנו קורדינטות מקומיות:
	\(u=(u^1,\dotsc,u^k): U \to \R^k\).
	
	\begin{example*}
		פרבולה
		\(P = \left\{(x,y) \mid y=x^2 \right\} \subset \R^2 \)
		עם קורדינטות מקומיות
		\(u: P \to \R\)
		הנתונות על ידי
		\(u(t,t^2)=t\).
	\end{example*}

	\begin{example*}
		ספירה
		\(S^2 = \left\{(x^1,x^2,x^3) \mid \sum (x^i)^2 = 1 \right\} \subset \R^3 \)
		עם קורדינטות מקומיות
		\(u=(u^1,u^2,u^3): S^2 \to \R^2\)
		הנתונת על ידי
		\(
			u^{-1}(\theta, \vphi) = \left(
				\cos\theta \sin\vphi,
				\sin\theta \sin\vphi,
				\cos\vphi
			\right)
		\).
	\end{example*}
	
	\begin{definition*}
		בכל נקודה
		$p\in M$
		יש הכלה של מרחבים משיקים
		\(\mrm{T}_p M \subset \mrm{T}_p \R^n\).
		בסיס עבור
		\(\mrm{T}_p \R^n\)
		הוא כזכור
		\(\frac{\prt}{\prt x^\alpha}\Big|_p\)
		וניתן להציג את הבסיס של
		\(\mrm{T}_p M\)
		על ידי
		\(\frac{\prt}{\prt u^i} = \sum \frac{\prt x^\alpha}{\prt u^i} \frac{\prt}{\prt x^\alpha}\).
		ניתן להסתכל על המרחב הוקטורי המאונך אליו, ולקבל את \textbf{המרחב הנורמלי} בנקודה:
		\[
			\mrm{N}_p M
			= \left(\mrm{T}_p M\right)^\bot
			= \left\{
				v \in \mrm{T}_p \R^n
				\mid
				\forall v \in \mrm{T}_p M: \langle v,w \rangle=0
			\right\}
		\]
		ונבחר (באופן קונסיסטנטי) בסיס אורתונורמלי עבורו
		\(w_{k+1}|_p,\dotsc,w_{n}|_p\),
		כלומר
		\(w_\mu=\sum w_\mu^\alpha \frac{\prt}{\prt x^\alpha}\).
		
		את כל המרחבים הנורמליים ניתן לאחד לכדי \textbf{האגד הנורמלי}:
		\[
			\mrm{N} M
			= \coprod_{p \in M} \mrm{N}_p M
			\subset \mrm{T} \R^n
			\cong \R^{2n}
		\]
		נזכור שיש לנו איזומורפיזם
		\(\mrm{T}_p M \cong \R^{n-k}\)
		וזה מאפשר להגדיר קורדינטות על האגד המשיק
		\((u,t): \mrm{N} M \to \R^n\)
		על ידי:
		\[
		(u,t)(p,v)
		= (u,t)(p,\sum v^\mu w_\mu)
		= (u(p),v^1,\dotsc,v^{n-k})
		\]
		לכן האגד הנורמלי מהווה יריעה בפני עצמו, ממימד \(n\).
	\end{definition*}

	\begin{example*}
		עבור
		\(p=(t,t^2)\)
		המרחב המשיק הוא
		\(
			\mrm{T}_p P
			= \spn\left\{
				\begin{pmatrix}
					1\\2t
				\end{pmatrix}
			\right\}
		\)
		ולכן המרחב הנורמלי הוא
		\(
			\mrm{N}_p P
			= \left\{
				a
				\begin{pmatrix}
					-2t\\1
				\end{pmatrix}
				\mid
				a \in \R
			\right\}
		\).
	\end{example*}

	\begin{example*}
		עבור
		\(p=(\cos\theta \sin\vphi, \sin\theta \sin\vphi, \cos\vphi)\)
		נחשב למשל:
		\[
			\left(\frac{\prt}{\prt u^1}\Big|_p\right)^1
			= \frac{\prt x^1}{\prt u^1}\Big|_p
			= \frac{\prt}{\prt \theta}\Big|_p (x^1 \circ u^{-1})
			= \frac{\prt}{\prt \theta}\Big|_p (\cos\theta \sin\vphi)
			= -\sin\theta \sin\vphi
		\]
		ובסך הכל המרחב המשיק הוא
		\(
			\mrm{T}_p S^2
			= \spn\left\{
				\begin{pmatrix}
					-\sin\theta \sin\vphi \\ \cos\theta \sin\vphi \\ 0
				\end{pmatrix},
				\begin{pmatrix}
					\cos\theta \cos\vphi \\ \sin\theta \cos\vphi \\ -\sin\vphi
				\end{pmatrix}
			\right\}
		\).
		\\
		לכן המרחב הנורמלי הוא
		\(
			\mrm{N}_p S^2
			= \left\{
				a
				\begin{pmatrix}
					\cos\theta \sin\vphi \\ \sin\theta \sin\vphi \\ \cos\vphi
				\end{pmatrix}
				\mid
				a \in \R
			\right\}
			= \left\{
				a
				\left(x(\theta, \vphi)\right)^T
				\mid
				a \in \R
			\right\}
		\).
	\end{example*}




	\section{נקודות מוקד}

	\begin{definition*}
		נזכור שיש לנו איזומורפיזם
		\(\mrm{T}_p \R^n \cong \R^n\)
		ששולח את הוקטור
		\(\frac{\prt}{\prt x^\alpha}\)
		לנקודה
		\(e^\alpha\).
		זה מאפשר לנו להגדיר העתקה
		\(E:\mrm{N} M \to \R^n\)
		שעבור נקודה
		\(p \in M \subset \R^n\)
		ווקטור
		\(v \in \mrm{N}_p M \subset \mrm{T}_p \R^n\)
		מוגדרת להיות
		\(E(p,v)=p+v\),
		או קצת יותר מפורשות:
		\[
			E\left(\sum p_\alpha e^\alpha,\sum v^\mu w_\mu\right)
			= E\left(\sum p_\alpha e^\alpha,\sum \sum v^\mu w_\mu^\alpha \frac{\prt}{\prt x^\alpha}\right)
			= \sum \left(p_\alpha + \sum v^\mu w_\mu^\alpha\right) e^\alpha,
		\]
		\todo{לא צריך לעשות משהו עם האינדקסים בסוף?}
		\\
		נקודה
		\(e \in \R^n\)
		נקראת \textbf{נקודת מוקד} מריבוי
		\(m>0\)
		של הנקודה
		\(p \in M\)
		אם יש וקטור
		\(v \in \mrm{N}_p M\)
		שעבורו
		\(e = E(p,v)\)
		והאפסיות של היעקוביאן בנקודה הוא \(m\), כלומר
		\(\dim \ker JE(p,v) = m\).
	\end{definition*}

	נבין מה הוא היעקוביאן בקורדינטות:
	\begin{equation*}\begin{split}
		JE^\alpha_i\Big|_p
		&= \frac{\prt (x^\alpha \circ E)}{\prt u^i}
		\\&= \frac{\prt}{\prt r^i}\left (x^\alpha \circ E \circ (u,t)^{-1}\left(r^1,\dotsc,r^n\right)\right)
		\\&= \frac{\prt}{\prt r^i} \left(x^\alpha \circ \left(u^{-1}(r^1,\dotsc,r^k) + \sum\sum r^\mu w_\mu^\beta(u^{-1}) e^\beta \right)\right)
		\\&= \frac{\prt}{\prt r^i} \left(x^\alpha \circ u^{-1}(r^1,\dotsc,r^k)\right) + \frac{\prt}{\prt r^i} \left(\sum r^\mu w_\mu^\alpha(u^{-1}) \right)
		\\&= \frac{\prt x^\alpha}{\prt u^i} + \sum t^\mu \frac{\prt w_\mu^\alpha}{\prt u^i}
	\end{split}\end{equation*}
	\begin{equation*}
		JE^\alpha_\mu\Big|_p
		= \frac{\prt (x^\alpha \circ E)}{\prt t^\mu}
		= \frac{\prt}{\prt r^\mu} \left(x^\alpha \circ u^{-1}(r^1,\dotsc,r^k)\right) + \frac{\prt}{\prt r^\mu} \left(\sum r^\nu w_\nu^\alpha(u^{-1}) \right)
		= w_\mu^\alpha
	\end{equation*}

	\begin{example*}
		בקורדינטות של
		\(\mrm{N} P\)
		שלקחנו מתקיים
		\(
			E(t,a) = (t-2at,t^2+a)
		\)
		ולכן
		\(
			JE = \begin{pmatrix}
				1-2a & -2t
				\\
				2t & 1
			\end{pmatrix}
		\).
		נחשב את הדטרמיננטה ונקבל שצריך להתקיים
		\(1-2a+4t^2=0\)
		כלומר
		\(a=\frac{1+4t^2}{2}\)
		ולכן הנקודה
		\(\left(4t^3,\frac{1+6t^2}{2}\right)\)
		נקודת מוקד מריבוי \(1\).
	\end{example*}

	\begin{example*}
		בקורדינטות של
		\(\mrm{N} S^2\)
		שלקחנו מתקיים
		\(
		E(\theta,\vphi,R) = (1+R)x(\theta,\vphi)
		\)
		ולכן:
		\[
			JE = \begin{pmatrix}
				-(1+R)(\sin\theta \sin\vphi) & (1+R)(\cos\theta \cos\vphi) & \cos\theta \sin\vphi
				\\
				(1+R)(\cos\theta \sin\vphi) & (1+R)(\sin\theta \cos\vphi) & \sin\theta \sin\vphi
				\\
				0 & 0 & \cos \vphi
			\end{pmatrix}
		\]
		נחשב את הדטרמיננטה ונקבל שצריך להתקיים
		\(-(1+R)^2 \cos^2\vphi \sin\vphi=0\)
		כלומר כאשר
		\(R=-1\)
		יש נקודת מוקד מריבוי \(2\).
	\end{example*}

	כעת נזכר במשפט סרד:
	\begin{theorem*}[משפט סרד]
		תהי
		\(f:M_1\to M_2\)
		העתקה
		\(C^1\)
		בין שתי יריעות מאותו מימד,
		אז קבוצת הערכים הסינגולריים היא ממידה \(0\).
	\end{theorem*}

	\begin{corollary}
		אוסף נקודות המוקד של \(M\) הוא ממידה \(0\).
	\end{corollary}




	\section{התבניות היסודיות ועקמומיות}
	
	\begin{definition*}
		תהי
		\(\gamma: I \to M\)
		מסילה.
		האורך שלה הוא
		\(\int_I \norm{\dot\gamma} \dd{t}\).
		מאחר והיא מסילה ב-\(M\), מתקיים
		\(\dot\gamma(t) \in \mrm{T}_{\gamma(t)} M\)
		אז ניתן להציג אותה על ידי
		\(\dot\gamma(t)=\sum \dot{\gamma}^i(t) \frac{\prt}{\prt u^i}\Big|_{\gamma(t)} \)
		ואז אפשר לרשום את האורך על ידי
		\(\int_I \sqrt{
			\sum \sum \dot{\gamma}^i \dot{\gamma}^j \left\langle \frac{\prt}{\prt u^i},\frac{\prt}{\prt u^j} \right\rangle
		} \dd{t}\).
		נבחין גם כי מתקיים:
		\[
			\left\langle \frac{\prt}{\prt u^i},\frac{\prt}{\prt u^j} \right\rangle
			=
			\left\langle
				\sum \frac{\prt x^\alpha}{\prt u^i} \frac{\prt}{\prt x^\alpha},
				\sum \frac{\prt x^\beta}{\prt u^j} \frac{\prt}{\prt x^\beta}
			\right\rangle
			=
			\sum \sum \frac{\prt x^\alpha}{\prt u^i} \frac{\prt x^\beta}{\prt u^j}
			\left\langle \frac{\prt}{\prt x^\alpha}, \frac{\prt}{\prt x^\beta} \right\rangle
			=
			\sum \frac{\prt x^\alpha}{\prt u^i} \frac{\prt x^\alpha}{\prt u^j}
		\]
		זה מגדיר
		\((0,2)\)
		שדה טנזורי שנקרא \textbf{התבנית היסודית הראשונה} (או המטריקה המושרית), ומסומן
		\(g_{ij}\).
		\\
		אנו מקבלים למשל שהאורך של המסילה הוא
		\(\int_I \sqrt{
			\sum \sum \dot{\gamma}^i \dot{\gamma}^j g_{ij}
		} \dd{t}\).
	\end{definition*}

	\begin{definition*}
		באופן דומה, נגדיר שדה טנזורי שנקרא \textbf{התבנית היסודית השנייה} על ידי
		\(\left(l_{ij}^\alpha\right) = \left(\frac{\prt^2 x^\alpha}{\prt u^i \prt u^j}\right)\).
		בהנתן וקטור יחידה נורמלי \(v\) אפשר לקחת להם מכפלה פנימית ולקבל את התבנית היסודית השנייה בכיוון הוקטור הנ"ל
		\(\left(\sum v^\alpha l_{ij}^\alpha\right)\).
	\end{definition*}

	\begin{definition*}
		בהנתן וקטור \(v\) כנ"ל,
		הע"ע של התבנית היסודית השנייה בכיוון הנ"ל,
		שמסומנים
		\(K_1,\dotsc,K_k\),
		נקראים \textbf{העקמומיויות הראשיות} בכיוון \(v\).
		ההפכיים שלהם,
		\(K_1^{-1},\dotsc,K_k^{-1}\),
		נקראים \textbf{רדיוסי העקמומיות הראשיים} בכיוון \(v\).
	\end{definition*}



	\section{נקודות מוקד ועקמומיות}
	
	\begin{theorem}
		יהיו
		\(q\in M\)
		נקודה וכן
		\(v\in\mrm{N}_q M\)
		וקטור יחידה.
		נקודות המוקד של \(q\) לאורך
		\(\ell = q+av\)
		הן הנקודות
		\(q + K_i^{-1} v\)
		(כאשר \(i\neq 0\)).
	\end{theorem}

	\begin{proof}
		\(q+av\)
		היא נקודת מוקד מריבוי \(m\) אם האפסיות של
		\(JE\)
		בנקודה היא \(m\).
		עלינו להביט באפסיות של
		\(
			\left(JE^\alpha_\beta\right)
			= \begin{pmatrix}
				\frac{\prt E^\alpha}{\prt u^i}
				\\ \frac{\prt E^\alpha}{\prt t^\mu}
			\end{pmatrix}
		\).
		מימד האפסיות לא משתנה על ידי כפל במטריצה הפיכה, אז נכפול במטריצה
		\(
			\begin{pmatrix}
				\frac{\prt x^\beta}{\prt u^j}
				\\ w^\beta_\nu
			\end{pmatrix}^T
		\)
		ונקבל:
		\begin{equation*}\begin{split}
			\begin{pmatrix}
				\sum \frac{\prt E^\alpha}{\prt u^i} \frac{\prt x^\alpha}{\prt u^j}
				& \sum \frac{\prt E^\alpha}{\prt u^i} w^\alpha_\nu
				\\
				\sum \frac{\prt E^\alpha}{\prt t^\mu} \frac{\prt x^\alpha}{\prt u^j}
				& \sum \frac{\prt E^\alpha}{\prt t^\mu} w^\alpha_\nu
			\end{pmatrix}
			&= 
			\begin{pmatrix}
				\sum \left(\frac{\prt x^\alpha}{\prt u^i} + \sum t^\mu \frac{\prt w_\mu^\alpha}{\prt u^i}\right) \frac{\prt x^\alpha}{\prt u^j}
				& \sum \left(\frac{\prt x^\alpha}{\prt u^i} + \sum t^\mu \frac{\prt w_\mu^\alpha}{\prt u^i}\right) w^\alpha_\nu
				\\
				\sum w^\alpha_\mu \frac{\prt x^\alpha}{\prt u^j}
				& \sum w^\alpha_\mu w^\alpha_\nu
			\end{pmatrix}
			\\&= 
			\begin{pmatrix}
				\sum \frac{\prt x^\alpha}{\prt u^i} \frac{\prt x^\alpha}{\prt u^j} + \sum t^\mu \sum \frac{\prt w_\mu^\alpha}{\prt u^i} \frac{\prt x^\alpha}{\prt u^j}
				& \sum t^\mu \sum \frac{\prt w_\mu^\alpha}{\prt u^i} w^\alpha_\nu
				\\
				0
				& \delta_{\mu\nu}
			\end{pmatrix}
		\end{split}\end{equation*}
		ומכיוון שהבלוק הימני התחתון הוא מטריצת היחידה, מספיק להתמקד בבלוק השמאלי העליון.
		כעת נזכור כי
		\(
			\sum w_\mu^\alpha \frac{\prt x^\alpha}{\prt u^j}
			=0
		\)
		ולכן
		\(
			0
			=\frac{\prt}{\prt u^i} \sum w_\mu^\alpha \frac{\prt x^\alpha}{\prt u^j}
			=\sum \frac{\prt w_\mu^\alpha}{\prt u^i} \frac{\prt x^\alpha}{\prt u^j} + \sum w_\mu^\alpha \frac{\prt^2 x^\alpha}{\prt u^i \prt u^j}
		\)
		ומכאן שהבלוק העליון הוא:
		\begin{equation*}\begin{split}
			\left(\sum \frac{\prt x^\alpha}{\prt u^i} \frac{\prt x^\alpha}{\prt u^j} + \sum t^\mu \sum \frac{\prt w_\mu^\alpha}{\prt u^i} \frac{\prt x^\alpha}{\prt u^j}\right)
			&= \left(g_{ij} - \sum t^\mu \sum w_\mu^\alpha l^\alpha_{ij}\right)
			\\&= \left(g_{ij} - \sum a v^\alpha l^\alpha_{ij}\right)
		\end{split}\end{equation*}
		נבחר את המטריקה כך שהיא תהיה מטריצת הזהות, קרי
		\(g_{ij}=\delta_{ij}\),
		ואפשר לכפול במספר שונה מאפס, אז מקבלים שזה האפסיות של:
		\[
			\left(a^{-1} \delta_{ij} - \sum v^\alpha l^\alpha_{ij}\right)
		\]
		כלומר הריבוי של הע"ע
		\(a^{-1}\)
		של
		\(
			\left(\sum v^\alpha l^\alpha_{ij}\right)
		\),
		כנדרש.
	\end{proof}
	

\end{document}