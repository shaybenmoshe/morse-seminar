\documentclass{article}
\usepackage{fontspec}
\usepackage{amsmath, amsthm, amssymb}
\usepackage[top=2cm,bottom=2cm,left=2.5cm,right=2cm]{geometry}

\usepackage{polyglossia}
\setdefaultlanguage{hebrew}
\setotherlanguages{english}

\newfontfamily\hebrewfont{David}

\setlength{\parindent}{0em}
\setlength{\parskip}{0.8em}

\newtheorem{theorem}{משפט}
\newtheorem*{theorem*}{משפט}

\theoremstyle{definition}
\newtheorem*{definition*}{הגדרה}
\newtheorem*{example*}{דוגמה}

\newcommand{\ncmd}{\newcommand}

\ncmd{\prt}{\partial}

\ncmd{\mbb}[1]{\mathbb{#1}}
\ncmd{\R}{\mbb{R}}
\ncmd{\mrm}[1]{\mathrm{#1}}

\ncmd{\vphi}{\varphi}

\DeclareMathOperator{\im}{Im}
\DeclareMathOperator{\spn}{span}

\ncmd{\inj}{\hookrightarrow}

\newcommand{\norm}[1]{\left\lVert#1\right\rVert}

\title{סמינר מורס – קיומן של פונקציות מורס}
\author{שי בן משה}
\date{21/11/2016}

\begin{document}
	\maketitle
	
	
	\section{הקדמה}
	
	מטרת ההרצאה הזו היא להוכיח את קיומן של פונקציות מורס על יריעה חלקה \(M\).
	
	לצורך כך, נזכר במשפט יסודי מהתורה של יריעות חלקות:
	\begin{theorem*}[משפט השיכון של וויטני]
		יש שיכון חלק
		\(M \inj \R^{2k}\).
	\end{theorem*}
	ולכן נוכל להניח מעתה כי
	\(M\subset \R^n\).
	
	\begin{definition*}
		עבור
		\(p\in \R^n\)
		נגדיר
		\(L_p:M\to \R\)
		על ידי
		\(L_p(q)=\norm{p-q}^2\).
	\end{definition*}

	נוכיח את המשפט הבא:
	\begin{theorem}
		לכל
		\(p\in \R^n\),
		פרט לקבוצה ממידה 0, \(L_p\) היא פונקציית מורס.
	\end{theorem}


	\section{גאומטריה דיפרנציאלית}
	
	כדי להגדיר דברים בקורדינטות, נניח שיש לנו קורדינטות מקומיות:
	\(x=(x^1,\dotsc,x^n): \R^k \to \R^n\).
	
	\begin{example*}
		פרבולה
		\(P = \left\{(x,y) \mid y=x^2 \right\} \subset \R^2 \)
		עם קורדינטות מקומיות
		\((x,y): \R \to \R^2\)
		הנתונות על ידי
		\(x(t)=t, y(t)=t^2\).
	\end{example*}

	\begin{example*}
		ספירה
		\(S^2 = \left\{(x^1,x^2,x^3) \mid \sum (x^i)^2 = 1 \right\} \subset \R^3 \)
		עם קורדינטות מקומיות
		\((x^1,x^2,x^3): \R^2 \to \R^3\)
		הנתונת על ידי
		\(
			x^1(\theta, \vphi) = \cos\theta \sin\vphi,
			x^2(\theta, \vphi) = \sin\theta \sin\vphi,
			x^3(\theta, \vphi) = \cos\vphi
		\).
	\end{example*}
	
	\begin{definition*}
		בכל נקודה
		$p\in M$
		יש הכלה של מרחבים משיקים
		\(\mrm{T}_p M \subset \mrm{T}_p \R^n\),
		והאוסף
		\(\frac{\prt x}{\prt u^1}\Big|_p,\dotsc,\frac{\prt x}{\prt u^k}\Big|_p\)
		מהווה בסיס עבורו.
		\\
		ניתן להסתכל על המרחב הוקטורי המאונך אליו, ולקבל את \textbf{המרחב הנורמלי} בנקודה:
		\[
			\mrm{N}_p M
			= \left(\mrm{T}_p M\right)^\bot
			= \left\{
				\vec{v} \in \mrm{T}_p \R^n
				\mid
				\forall \vec{w} \in \mrm{T}_p M: \langle \vec{v},\vec{w} \rangle=0
			\right\}
		\]
		ונבחר בסיס אורתונורמלי עבורו
		\(\vec{w}_1|_p,\dotsc,\vec{w}_{n-k}|_p\).
		(באופן קונסיסטנטי)
		\\
		את כל המרחבים הנורמליים ניתן לאחד לכדי \textbf{האגד הנורמלי}:
		\[
			\mrm{N} M
			= \coprod_{p \in M} \mrm{N}_p M
			\subset \mrm{T} \R^n
			\cong \R^{2n}
		\]
		האגד הנורמלי מהווה יריעה בפני עצמו, ומימדו הוא \(n\), שכן המימד של כל מרחב נורמלי הוא \(n-k\).
	\end{definition*}

	\begin{example*}
		עבור
		\(p=(t,t^2)\)
		המרחב המשיק הוא
		\(
			\mrm{T}_p P
			= \spn\left\{
				\begin{pmatrix}
					1\\2t
				\end{pmatrix}
			\right\}
		\)
		ולכן המרחב הנורמלי הוא
		\(
			\mrm{N}_p P
			= \left\{
				a
				\begin{pmatrix}
					-2t\\1
				\end{pmatrix}
				\mid
				a \in \R
			\right\}
		\).
	\end{example*}

	\begin{example*}
		עבור
		\(p=(\cos\theta \sin\vphi, \sin\theta \sin\vphi, \cos\vphi)\)
		המרחב המשיק הוא
		\(
			\mrm{T}_p S^2
			= \spn\left\{
				\begin{pmatrix}
					-\sin\theta \sin\vphi \\ \cos\theta \sin\vphi \\ 0
				\end{pmatrix},
				\begin{pmatrix}
					\cos\theta \cos\vphi \\ \sin\theta \cos\vphi \\ -\sin\vphi
				\end{pmatrix}
			\right\}
		\)
		ולכן המרחב הנורמלי הוא
		\(
			\mrm{N}_p S^2
			= \left\{
				a
				\begin{pmatrix}
					\cos\theta \sin\vphi \\ \sin\theta \sin\vphi \\ \cos\vphi
				\end{pmatrix}
				\mid
				a \in \R
			\right\}
		\).
	\end{example*}

	\begin{definition*}
		האיזומורפיזם
		\(\mrm{T}_p \R^n \cong \R^n\)
		מאפשר לנו להגדיר העתקה
		\(E:\mrm{N} M \to \R^n\)
		שעבור נקודה
		\(p \in M \subset \R^n\)
		ווקטור
		\(\vec{v} \in \mrm{N}_p M\)
		מוגדרת להיות
		\(E(p,\vec{v})=p+v\).
		\\
		נקודה
		\(e \in \R^n\)
		נקראת \textbf{נקודת מוקד} מריבוי
		\(\mu>0\)
		בנקודה
		\(p \in M\)
		אם יש ווקטור
		\(\vec{v} \in \mrm{N}_p M\)
		שעבורו
		\(e = E(p,\vec{v})\)
		ומימד הגרעין של היעקוביאן בנקודה הוא \(\mu\), כלומר
		\(\dim \ker JE(p,\vec{v}) = \mu\).
	\end{definition*}

	\begin{example*}
		בקורדינטות של
		\(\mrm{N} P\)
		שלקחנו מתקיים
		\(
			E(t,a) = (t-2at,t^2+a)
		\)
		ולכן
		\(
			{JE}^i_j = \begin{pmatrix}
				1-2a & -2t
				\\
				2t & 1
			\end{pmatrix}
		\)
	\end{example*}

\end{document}