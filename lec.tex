\documentclass{article}
\usepackage{fontspec}
\usepackage{amsmath, amsthm, amssymb}
\usepackage[top=2cm,bottom=2cm,left=2.5cm,right=2cm]{geometry}

\usepackage{polyglossia}
\setdefaultlanguage{hebrew}
\setotherlanguages{english}

\newfontfamily\hebrewfont{David}

\setlength{\parindent}{0em}
\setlength{\parskip}{0.8em}

\newtheorem{theorem}{משפט}
\newtheorem{lemma}{למה}
\newtheorem{corollary}{מסקנה}
\newtheorem*{theorem*}{משפט}

\theoremstyle{definition}
\newtheorem*{definition*}{הגדרה}
\newtheorem*{example*}{דוגמה}

\newcommand{\ncmd}{\newcommand}

%\ncmd{\todo}[1]{}
\ncmd{\todo}[1]{\textbf{TODO #1}}

\ncmd{\prt}{\partial}

\ncmd{\mbb}[1]{\mathbb{#1}}
\ncmd{\R}{\mbb{R}}
\ncmd{\mrm}[1]{\mathrm{#1}}
\ncmd{\dd}[1]{\mrm{d}#1}
\ncmd{\diff}[1]{\mrm{d}#1}

\ncmd{\vphi}{\varphi}

\ncmd{\lip}{\left\langle}
\ncmd{\rip}{\right\rangle}

\DeclareMathOperator{\im}{Im}
\DeclareMathOperator{\spn}{span}

\ncmd{\inj}{\hookrightarrow}

\newcommand{\norm}[1]{\left\lVert#1\right\rVert}

\title{סמינר מורס – קיומן של פונקציות מורס}
\author{שי בן משה}
\date{21/11/2016}

\begin{document}
	\maketitle
	
	
	
	
	\section{הקדמה}
	
	מטרת ההרצאה הזו היא להוכיח את קיומן של פונקציות מורס על יריעה חלקה \(M\) ממימד \(k\).
	
	לצורך כך, נזכר במשפט יסודי מהתורה של יריעות חלקות:
	\begin{theorem*}[משפט השיכון של וויטני]
		קיים \(n\) כך שקיים שיכון חלק
		\(\psi: M \inj \R^n\).
	\end{theorem*}
	\begin{proof}
		נוכיח רק עבור \(M\) קומפקטית.
		ובכן, יהי
		\(\left\{\left(U_i,\varphi_i\right)\right\}_{i=1}^m\)
		כיסוי סופי, שקיים מקומפקטיות.
		נבחר חלוקת יחידה
		\(\left\{\rho_i\right\}_{i=1}^m\)
		שמתאימה לכיסוי הנ"ל.
		נרחיב את ההעתקות שלנו להעתקות
		\(\tilde{\varphi}_i: M \to \R^{k+1}\)
		שמוגדרות
		\(\tilde{\varphi}_i(p) = \left(\varphi_i(p)\rho_i(p),\rho_i(p)\right)\)
		עבור
		\(p \in U_i\)
		ו-\(0\) אחרת.
		כעת נגדיר
		\(\psi: M \to \R^{(k+1)m}\)
		על ידי
		\(
			\psi(p)=\left(\tilde{\varphi}_1(p),\dotsc,\tilde{\varphi}_m(p)\right)
		\).
		\\
		נוכיח שהיא חח"ע, אם
		\(\psi(p)=\psi(q)\)
		אז יש \(i\) כך שמתקיים
		\(\tilde{\varphi}_i(p)=\tilde{\varphi}_i(p)\neq 0\)
		ובפרט,
		\(\varphi_i(p)=\varphi_i(p)\neq 0,\rho_i(p)=\rho_i(q)\neq 0\)
		ומהחלק השני
		\(p,q \in U_i\)
		והקורדינטות המקומיות הפיכות שם, אז
		\(p=q\)
		מה שמוכיח חח"ע.
		\\
		כדי לראות שהדיפרנציאל חח"ע, נבחין כי
		\(\diff{\psi}=\left(\diff{\tilde{\varphi_1}},\dotsc,\diff{\tilde{\varphi_m}}\right)\)
		אז עבור נקודה
		\(p\in U_i\)
		מספיק לבדוק שהדיפרנציאל
		\(\diff{\tilde{\varphi_i}}\)
		הוא חח"ע.
		הוא מקיים
		\(
			\diff{\tilde{\varphi_i}}|_p(v)
			=\left(
				\diff{\varphi_i}(v)\rho_i(p) + \varphi_i(p)\diff{\rho_i(v)},
				\diff{\rho_i}(v)
			\right)
		\),
		אם
		\(\diff{\rho_i}(v)\neq 0\)
		אז סיימנו, אחרת החלק הראשון הוא
		\(\diff{\varphi_i}(v)\rho_i(p)\)
		ומאחר ו-\(p\in U_i\) אז
		\(\rho_i(p)=0\)
		ומאחר ו-\(\diff{\varphi_i}\) חח"ע, סיימנו.
		\\
		לבסוף, נראה שצמצום הטווח לתמונה מקיים שההפכית אכן רציפה,
		קרי שהמקור של סגורות הוא סגור.
		זה שקול לכך ש-\(\psi\) סגורה,
		אבל מאחר ו-\(M\) קומפקטית, אז סגורה היא גם קומפקטית,
		ולכן תמונתה קומפקטית ולכן גם סגורה, כנדרש.
	\end{proof}
	מעתה נניח כי
	\(M \subset \R^n\)
	
	\begin{definition*}
		עבור
		\(p\in \R^n\)
		נגדיר
		\(L_p:M\to \R\)
		על ידי
		\(L_p(q)=\frac{1}{2}\norm{q-p}^2\).
	\end{definition*}

	נוכיח את המשפט הבא:
	\begin{theorem}\label{main-thm}
		לכל
		\(p\in \R^n\),
		פרט לקבוצה ממידה 0, \(L_p\) היא פונקציית מורס.
	\end{theorem}

	כדי להגדיר דברים בקורדינטות, נניח שיש לנו קורדינטות מקומיות:
	\(\varphi=(u^1,\dotsc,u^k): U \to \R^k\).
	
	\begin{example*}
		קו ישר שמשוכן כפרבולה, קרי
		\(M = \left\{ (t,t^2)\mid t \in \R \right\}\)
		עם קורדינטות
		\(\varphi=(u^1): M \to \R, u^1(t,t^2)=t\).
	\end{example*}
	
	\begin{example*}
		ספירה
		\(S^2 = \left\{(x^1,x^2,x^3) \mid \sum (x^i)^2 = 1 \right\} \subset \R^3 \)
		שמשוכנת כרגיל עם קורדינטות מקומיות
		\(\varphi=(u^1,u^2): S^2 \to \R^2\)
		הנתונת על ידי
		\(
			\varphi^{-1}(\theta, \vphi) = \left(
				\cos\theta \sin\vphi,
				\sin\theta \sin\vphi,
				\cos\vphi
			\right)
		\).
	\end{example*}




	\section{מרחב נורמלי}
	
	\begin{definition*}
		בכל נקודה
		$p\in M$
		יש שיכון
		\(\mrm{T}_p M \subset \mrm{T}_p \R^n\).
		בסיס עבור
		\(\mrm{T}_p M\)
		הוא
		\(w_i|_p=\frac{\prt}{\prt u^i}\Big|_p\).
		ניתן להסתכל על המרחב הוקטורי המאונך, ולקבל את \textbf{המרחב הנורמלי} בנקודה:
		\[
			\mrm{N}_p M
			= \left(\mrm{T}_p M\right)^\bot
			= \left\{
				w \in \mrm{T}_p \R^n
				\mid
				\forall v \in \mrm{T}_p M: \langle w,v \rangle=0
			\right\}
		\]
		ונבחר (באופן קונסיסטנטי) בסיס אורתונורמלי עבורו
		\(w_{k+1}|_p,\dotsc,w_{n}|_p\),
		ונציג אותו לפי הבסיס הסטנדרטי:
		\(w_\mu = \sum w^\alpha_\mu \frac{\prt}{\prt x^\alpha}\).
		
		את כל המרחבים הנורמליים ניתן לאחד לכדי \textbf{האגד הנורמלי}:
		\[
			\mrm{N} M
			= \coprod_{p \in M} \mrm{N}_p M
			\subset \mrm{T} \R^n
			\cong \R^{2n}
		\]
		בחירת הבסיס מאפשרת לנו להגדיר קורדינטות על המרחב הנורמלי
		\((\varphi,\phi): \mrm{N} M \to \R^n\)
		על ידי:
		\[
			(\varphi,\phi)(p,v)
			= (\varphi,\phi)(p,\sum v^\mu w_\mu)
			= (\varphi(p),v^1,\dotsc,v^{n-k})
		\]
		לכן האגד הנורמלי מהווה יריעה בפני עצמו, ממימד \(n\).
	\end{definition*}

	\begin{example*}
		עבור
		\(p=t\)
		המרחב המשיק, לפי הקורדינטות, נפרש על ידי
		\(\frac{\prt}{\prt u^1}\),
		ונחשב אותו:
		\[
			\frac{\prt}{\prt u^1}\Big|_{(t,t^2)} f
			=\frac{\prt}{\prt r^1}\Big|_t\left(f\circ\varphi^{-1}\left(r^1\right)\right)
			=\frac{\prt}{\prt r^1}\Big|_t\left(f\left(r^1,\left(r^1\right)^2\right)\right)
			=\frac{\prt f}{\prt x^1}\Big|_{(t,t^2)}+2t\frac{\prt f}{\prt x^2}\Big|_{(t,t^2)}
		\]
		כלומר
		\(
			\frac{\prt}{\prt u^1}\Big|_{(t,t^2)}
			=\frac{\prt}{\prt x^1}\Big|_{(t,t^2)}+2t\frac{\prt}{\prt x^2}\Big|_{(t,t^2)}
		\)
		ולכן המרחב הנורמלי הוא:
		\[
			\mrm{N}_t M
			= \left\{
				-2at\frac{\prt}{\prt x^1}\Big|_{(t,t^2)}+a\frac{\prt}{\prt x^2}\Big|_{(t,t^2)}
				\mid
				a \in \R
			\right\}
		\]
		וקורדינטות למרחב הנורמלי הן:
		\[
			(u^1,u^2)
			=\left(\varphi,\phi\right)\left(t,-2at\frac{\prt}{\prt x^1}\Big|_{(t,t^2)}+a\frac{\prt}{\prt x^2}\Big|_{(t,t^2)}\right)
			=(t,a)
		\]
	\end{example*}

	\begin{example*}
		עבור
		\(p=(\cos\theta \sin\vphi, \sin\theta \sin\vphi, \cos\vphi)\)
		נחשב למשל:
		\[
			\left(\frac{\prt}{\prt u^1}\Big|_p\right)^1
			= \frac{\prt x^1}{\prt u^1}\Big|_p
			= \frac{\prt}{\prt \theta}\Big|_p (x^1 \circ u^{-1})
			= \frac{\prt}{\prt \theta}\Big|_p (\cos\theta \sin\vphi)
			= -\sin\theta \sin\vphi
		\]
		ובסך הכל המרחב המשיק הוא
		\(
			\mrm{T}_p S^2
			= \spn\left\{
				\begin{pmatrix}
					-\sin\theta \sin\vphi \\ \cos\theta \sin\vphi \\ 0
				\end{pmatrix},
				\begin{pmatrix}
					\cos\theta \cos\vphi \\ \sin\theta \cos\vphi \\ -\sin\vphi
				\end{pmatrix}
			\right\}
		\).
		\\
		לכן המרחב הנורמלי הוא
		\(
			\mrm{N}_p S^2
			= \left\{
				a
				\begin{pmatrix}
					\cos\theta \sin\vphi \\ \sin\theta \sin\vphi \\ \cos\vphi
				\end{pmatrix}
				\mid
				a \in \R
			\right\}
			= \left\{
				a
				\left(x(\theta, \vphi)\right)^T
				\mid
				a \in \R
			\right\}
		\).
	\end{example*}




	\section{נקודות מוקד}

	\begin{definition*}
		נזכור שעבור
		\(p\in\R^n\)
		יש לנו איזומורפיזם
		\(D_p: \R^n \to \mrm{T}_p \R^n\)
		ולכן זה מגדיר איזומורפיזם
		\(D: \R^n \times \R^n \to \mrm{T} \R^n\)
		על ידי
		\(D(p,v)=D_p(v)\).
		זה גם מגדיר לנו העתקה
		\(\tilde{D}^{-1}: \mrm{T} \R^n \to \R^n\)
		ששולחת לנקודה שגזירה בכיוונה זה הוקטור עצמו.
		מאחר ולפי הגדרה
		\(\mrm{N} M \subset \mrm{T} \R^n\)
		זה מאפשר לנו להגדיר העתקה
		\(E:\mrm{N} M \to \R^n\)
		שעבור נקודה
		\(p \in M \subset \R^n\)
		ווקטור
		\(v \in \mrm{N}_p M\)
		מוגדרת להיות
		\(E(p,v)=p+\tilde{D}^{-1}(v)\),
		או קצת יותר מפורשות:
		\[
			E\left(p,v\right)
			=E\left(\sum p_\alpha e^\alpha,\sum \sum v^\mu w_\mu^\alpha(p) \frac{\prt}{\prt x^\alpha}\right)
			=\sum p_\alpha e^\alpha + \sum\sum v^\mu w_\mu^\alpha(p) e^\alpha
			=\sum \left(p_\alpha + \sum v^\mu w_\mu^\alpha(p)\right) e^\alpha
		\]
		ובקורדינטות:
		\[
			E\left(\left(\varphi,\phi\right)^{-1}\left(r^1,\dotsc,r^n\right)\right)
			=\sum \left(
				x^\alpha\left(\varphi^{-1}\right)
				+\sum r^\mu w_\mu^\alpha\left(\varphi^{-1}\right)
			\right) e^\alpha
		\]
		נקודה
		\(e \in \R^n\)
		נקראת \textbf{נקודת מוקד} מריבוי
		\(m>0\)
		של הנקודה
		\(p \in M\)
		אם יש וקטור
		\(v \in \mrm{N}_p M\)
		שעבורו
		\(e = E(p,v)\)
		והאפסיות של הדיפרנציאל בנקודה הוא \(m\), כלומר
		\(\mrm{null}\left(\diff{E}|_{(p,v)}\right) = \dim \ker \diff{E}|_{(p,v)} = m\).
	\end{definition*}

	עבור נקודה
	\((q,v)\)
	יש איזומורפיזם קנוני:
	\[
		\mrm{T}_{(p,v)} \mrm{N} M
		\cong \mrm{T}_q M \oplus \mrm{T}_v \mrm{N}_p M
		\cong \mrm{T}_p M \oplus \mrm{N}_p M
		= \mrm{T}_p \R^n
	\]
	ומאחר והעתקה
	\(E: \mrm{N} M \to \R^n\)
	אז באופן טבעי מקבלים שהדיפרנציאל הוא
	\(\diff{E}: \mrm{T}_p M \oplus \mrm{N}_p M \to \mrm{T}_p M \oplus \mrm{N}_p M\):
	\begin{equation*}\begin{split}
		\diff{E}^\alpha_i|_{\left(p,v\right)}
		&= \frac{\prt E^\alpha}{\prt u^i}
		\\&= \frac{\prt}{\prt r^i}\left (E^\alpha\left((\varphi,\phi)^{-1}(r)\right)\right)
		\\&= \frac{\prt}{\prt r^i} \left(x^\alpha\left(\varphi^{-1}\right)
		+\sum r^\mu w_\mu^\alpha\left(\varphi^{-1}\right)\right)
		\\&= \frac{\prt}{\prt u^i}(x^\alpha) + \sum t^\mu \frac{\prt}{\prt u^i} \left(w_\mu^\alpha\right)
	\end{split}\end{equation*}
	ולכן עבור
	\(X\in \mrm{T}_p M\):
	\[
		\diff{E}|_{\left(p,v\right)} \left(X\right)
		=X+X\tilde{v}
	\]
	ובדומה
	\begin{equation*}
		\diff{E}^\alpha_\mu|_{\left(p,v\right)}
		= \frac{\prt E^\alpha}{\prt u^\mu}
		= \frac{\prt}{\prt r^\mu} \left(x^\alpha\left(\varphi^{-1}\right)
		+\sum r^\nu w_\nu^\alpha\left(\varphi^{-1}\right)\right)
		= w_\mu^\alpha|_p
	\end{equation*}
	ולכן עבור
	\(X\in \mrm{T}_p M\):
	\[
		\diff{E}|_{\left(p,v\right)} \left(X\right) = X
	\]

	\begin{example*}
		בקורדינטות של
		\(\mrm{N} M\)
		שלקחנו מתקיים
		\(
			E(\left(\varphi,\phi\right)^{-1}(t,a))
			=(t,t^2)+(-2at,a)
			=(t-2at,t^2+a)
		\),
		נחשב את הדיפרנציאל בקורדינטות האלו:
		\begin{equation*}\begin{split}
			\diff{E}\left(\frac{\prt}{\prt u^1}\right)
			&=\frac{\prt E^1}{\prt u^1} \frac{\prt}{\prt x^1}
			+\frac{\prt E^2}{\prt u^1} \frac{\prt}{\prt x^2}
			\\&=\frac{\prt}{\prt r^1}\left(E^1\left(\left(\varphi,\phi\right)^{-1}\left(r^1,r^2\right)\right)\right) \frac{\prt}{\prt x^1}
			+\frac{\prt}{\prt r^1}\left(E^2\left(\left(\varphi,\phi\right)^{-1}\left(r^1,r^2\right)\right)\right) \frac{\prt}{\prt x^2}
			\\&=\frac{\prt}{\prt r^1}\left(r^1-2r^2 r^1\right) \frac{\prt}{\prt x^1}
			+\frac{\prt}{\prt r^1}\left(\left(r^1\right)^2+r^2\right) \frac{\prt}{\prt x^2}
			\\&=\left(1-2a\right) \frac{\prt}{\prt x^1}
			+2t \frac{\prt}{\prt x^2}
		\end{split}\end{equation*}
		ובאופן דומה
		\[
			\diff{E}\left(\frac{\prt}{\prt u^2}\right)
			=-2t \frac{\prt}{\prt x^1}
			+\frac{\prt}{\prt x^2}
		\]
		או כמטריצה
		\(
			\diff{E}
			=\begin{pmatrix}
				1-2a & -2t
				\\
				2t & 1
			\end{pmatrix}
		\).
		נחשב את הדטרמיננטה, כדי לראות איפה היא מנוונת, ונקבל שצריך להתקיים
		\(1-2a+4t^2=0\)
		כלומר
		\(a=\frac{1+4t^2}{2}\)
		ולכן הנקודה
		\(
			E(\left(\varphi,\phi\right)^{-1}(t,\frac{1+4t^2}{2}))
			=\left(-4t^3,\frac{1+6t^2}{2}\right)
		\)
		נקודת מוקד מריבוי \(1\).
	\end{example*}

	\begin{example*}
		בקורדינטות של
		\(\mrm{N} S^2\)
		שלקחנו מתקיים
		\(
		E(\theta,\vphi,R) = (1+R)x(\theta,\vphi)
		\)
		ולכן:
		\[
			JE = \begin{pmatrix}
				-(1+R)(\sin\theta \sin\vphi) & (1+R)(\cos\theta \cos\vphi) & \cos\theta \sin\vphi
				\\
				(1+R)(\cos\theta \sin\vphi) & (1+R)(\sin\theta \cos\vphi) & \sin\theta \sin\vphi
				\\
				0 & 0 & \cos \vphi
			\end{pmatrix}
		\]
		נחשב את הדטרמיננטה ונקבל שצריך להתקיים
		\(-(1+R)^2 \cos^2\vphi \sin\vphi=0\)
		כלומר כאשר
		\(R=-1\)
		יש נקודת מוקד מריבוי \(2\).
	\end{example*}

	כעת נזכר במשפט סרד:
	\begin{theorem*}[משפט סרד]
		תהי
		\(f:M_1\to M_2\)
		העתקה
		\(C^1\)
		בין שתי יריעות מאותו מימד,
		אז קבוצת הערכים הסינגולריים של \(f\)
		(כלומר, נקודות \(y=f(x)\) ב-\(M_2\)שבהן \(\diff{f}\) סינגולרית)
		היא ממידה \(0\).
	\end{theorem*}

	\begin{corollary}\label{measure-zero}
		אוסף נקודות המוקד של \(M\) הוא ממידה \(0\).
	\end{corollary}




	\section{לקראת המשפט המרכזי}
	\begin{example*}
		נחשב מתי
		\(f=L_p\)
		היא פונקציית מורס, כאשר
		\(p=(p_1,p_2)\in \R^2\).
		ובכן,
		\[
			f(t,t^2)
			=\frac{1}{2}\norm{(t,t^2)-p}^2
			=\frac{1}{2}\left(t^2-2tp_1+p_1^2+t^4-2t^2p_2+p_2^2\right)
		\]
		אז הדיפרנציאל מיוצג על ידי:
		\begin{equation*}\begin{split}
			\frac{\prt f}{\prt u^1}
			&=\frac{\prt}{\prt r^1}\left(f\circ\left(\varphi,\phi\right)^{-1}\left(r^1\right)\right)
			\\&=\frac{\prt}{\prt r^1}\left(f\left(r^1,\left(r^1\right)^2\right)\right)
			\\&=\frac{\prt}{\prt r^1}\left(\left(r^1\right)^2-2r^1 p_1+p_1^2+\left(r^1\right)^4-2\left(r^1\right)^2p_2+p_2^2\right)
			\\&=\frac{1}{2}\left(2t-2p_1+4t^3-4tp_2\right)
			\\&=t-p_1+2t^3-2tp_2
			\\&=\lip
				\begin{pmatrix}
					1 \\ 2t
				\end{pmatrix},
				\begin{pmatrix}
					t \\ t^2
				\end{pmatrix}
				-\begin{pmatrix}
				p_1 \\ p_2
				\end{pmatrix}
			\rip
		\end{split}\end{equation*}
		ואם הדיפרנציאל מתאפס בנקודה, אז ההסיאן הוא:
		\[
			\frac{\prt^2 f}{\prt \left(u^1\right)^2}
			=1+6t^2-2p_2
		\]
		נדרוש ששני הגדלים לעיל יתאפסו, מהשני נקבל
		\(p_2=\frac{1+6t^2}{2}\)
		ולכן מהראשון נקבל
		\(p_1=t+2t^3-t-6t^3=-4t^3\),
		כלומר שמתקיים
		\(p=\left(-4t^3,\frac{1+6t^2}{2}\right)\)
		אם ורק אם לפונקציה יש סינגולריות מנוונת, שזה אם ורק אם היא לא מורס,
		בדיוק מה שקיבלנו על נקודות המוקד!
	\end{example*}




	\section{עוד על נקודות מוקד}
	
	\begin{theorem}\label{focal-curvature}
		יהיו
		\(q\in M\)
		נקודה וכן
		\(v\in\mrm{N}_q M\)
		וקטור יחידה.
		נקודות המוקד של \(q\) לאורך
		\(\ell = q+D^{-1}(av)\)
		הן הנקודות
		\(q + D^{-1}(K_i^{-1} v)\)
		(כאשר \(K_i\neq 0\)).
	\end{theorem}

	\begin{proof}
		עבור נקודה
		\((q,v)\)
		נגדיר את ההעתקה
		\(A|_{(q,v)}(X,Y)=\lip \diff{E}|_{(q,v)}(X),Y \rip\).
		מאחר ואנחנו כופלים פנימית במשהו מדרגה מלאה, האפסיות נשארת זהה, קרי
		\(
			\mrm{null}\left(\diff{E}|_{(q,v)}\right)
			=\mrm{null}\left(A|_{(q,v)}\right)
		\).
		נפרק את \(X,Y\) לחלקים המשיקים והמאונכים ונקבל:
		\begin{equation*}\begin{split}
			A|_{(q,v)}(X^T+X^N,Y^T+Y^N)
			&= \lip \diff{E}(X^T+X^N),Y^T+Y^N \rip
			\\&= \lip X^T+X^T\tilde{v} +X^N ,Y^T+Y^N \rip
			\\&= \lip X^T,Y^T \rip + \lip X^T\tilde{v},Y^T \rip
			+ \lip X^T\tilde{v},Y^N \rip + \lip X^N,Y^N \rip
		\end{split}\end{equation*}
		נסמן את הצמצום למרחב
		\(\mrm{T}_q M \otimes \mrm{T}_q M\)
		על ידי
		\(B|_{(q,v)}: \mrm{T}_q \otimes \mrm{T}_q M \to \R\)
		המוגדרת
		\(B|_{(q,v)}(X,Y) = \lip X,Y \rip + \lip X\tilde{v},Y \rip\).
		נזכור גם כי
		\(\lip v,Y \rip=0\)
		ולכן
		\(
			\lip X\tilde{v},Y \rip + \lip v,X\tilde{Y} \rip=0
		\)
		אז אפשר לרשום
		\(B|_{(q,v)}(X,Y) = \lip X,Y \rip - \lip v,X\tilde{Y} \rip\).
		הצמצום למרחב
		\(\mrm{N}_q M \otimes \mrm{N}_q M\)
		נותן העתקה מדרגה מלאה, כי זו מכפלה פנימית בתוך
		\(\R^n\)
		ולכן האפסיות שווה לאפסיות של הצמצום הקודם, כלומר
		\(
			\mrm{null}\left(\diff{E}|_{(q,v)}\right)
			=\mrm{null}\left(B|_{(q,v)}\right)
		\).
	\end{proof}




	\section{המשפט המרכזי}
	
	כעת נוכיח את המשפט המרכזי, משפט \ref{main-thm}.
	\begin{proof}
		תהי
		\(p\in\R^n\)
		ונסמן
		\(v(q) = D_q(p-q) \in \mrm{N}_q M\).
		אז מתקיים:
		\[
			f(q)
			=L_p(q)
			=\frac{1}{2}\norm{p-q}^2
			=\frac{1}{2}\lip p-q,p-q \rip
		\]
		נחשב את הדיפרנציאל
		\(\diff{f}\)
		בנקודה:
		\begin{equation*}\begin{split}
			\diff{f}\left(\frac{\prt}{\prt u^i}\right)
			&=\frac{\prt}{\prt u^i}(f)
			\\&=\frac{1}{2}\frac{\prt}{\prt r^i}\left(\lip p-\varphi^{-1},p-\varphi^{-1} \rip\right)
			\\&=\frac{1}{2}\frac{\prt}{\prt r^i}\left(\sum p_\alpha^2+\left(\varphi_\alpha^{-1}\right)^2-p_\alpha\varphi_\alpha^{-1}\right)
			\\&=\sum (q_\alpha-p_\alpha)\frac{\prt x^\alpha}{\prt u^i}
			\\&=-\lip v(q),\frac{\prt}{\prt u^i} \rip
		\end{split}\end{equation*}
		ולכן באופן כללי:
		\[
			\diff{f}\left(X\right)
			=-\lip v(q),X \rip \frac{\dd}{\dd x}\Big|_{f(q)}
		\]
		ומכאן שהנקודה \(q\) היא קריטית אם ורק אם
		\(v(q)\in \mrm{N}_p M\).
		\\
		כעת נחשב את ההסיאן בנקודה קריטית, ונקבל בדומה:
		\begin{equation*}\begin{split}
			\mrm{Hess} f\Big|_q\left(\frac{\prt}{\prt u^i},\frac{\prt}{\prt u^j}\right)
			&=\frac{\prt}{\prt u^j}\left(\sum (q_\alpha-p_\alpha)\frac{\prt x^\alpha}{\prt u^i}\right)
			\\&=\sum \frac{\prt x^\alpha}{\prt u^j}\frac{\prt x^\alpha}{\prt u^i}
			+ \sum (q_\alpha-p_\alpha)\frac{\prt^2 x^\alpha}{\prt u^j\prt u^i}
			\\&=\lip \frac{\prt x}{\prt u^i},\frac{\prt x}{\prt u^j} \rip
			- \lip v(q),\frac{\prt^2}{\prt u^i\prt u^j} \rip
		\end{split}\end{equation*}
		ולכן באופן כללי:
		\[
			\mrm{Hess} f\Big|_q\left(X,Y\right)
			=\lip X,Y \rip - \lip v(q),X\tilde{Y} \rip
		\]
		
		אם כן, \(f\) מנוונת, כלומר ל-\(f\) יש נקודה קריטית מנוונת, אם יש
		\(q \in M\)
		כך ש-\(p\) היא נקודה מוקד יחד עם הוקטור
		\(v(q)\),
		ובפרט \(p\) צריכה להיות נקודת מוקד.
		ואולם, יש רק קבוצה ממידה \(0\) של נקודות מוקד, וסיימנו.
	\end{proof}
	

\end{document}