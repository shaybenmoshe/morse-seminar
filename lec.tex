\documentclass{article}
\usepackage{fontspec}
\usepackage{amsmath, amsthm, amssymb}
\usepackage[top=2cm,bottom=2cm,left=2.5cm,right=2cm]{geometry}

\usepackage{polyglossia}
\setdefaultlanguage{hebrew}
\setotherlanguages{english}

\newfontfamily\hebrewfont{David}

\setlength{\parindent}{0em}
\setlength{\parskip}{0.5em}

\newtheorem{theorem}{משפט}
\newtheorem*{theorem*}{משפט}

\theoremstyle{definition}
\newtheorem*{definition*}{הגדרה}

\newcommand{\ncmd}{\newcommand}

\ncmd{\mbb}[1]{\mathbb{#1}}
\ncmd{\R}{\mbb{R}}
\ncmd{\mrm}[1]{\mathrm{#1}}

\ncmd{\inj}{\hookrightarrow}

\newcommand{\norm}[1]{\left\lVert#1\right\rVert}

\title{סמינר מורס – קיומן של פונקציות מורס}
\author{שי בן משה}
\date{21/11/2016}

\begin{document}
	\maketitle
	
	
	\section{הקדמה}
	
	מטרת ההרצאה הזו היא להוכיח את קיומן של פונקציות מורס על יריעה חלקה \(M\).
	
	לצורך כך, נזכר במשפט יסודי מהתורה של יריעות חלקות:
	\begin{theorem*}[משפט השיכון של וויטני]
		יש שיכון חלק
		\(M \inj \R^{2k}\).
	\end{theorem*}
	ולכן נוכל להניח מעתה כי
	\(M\subset \R^n\).
	
	\begin{definition*}
		עבור
		\(p\in \R^n\)
		נגדיר
		\(L_p:M\to \R\)
		על ידי
		\(L_p(q)=\norm{p-q}^2\).
	\end{definition*}

	נוכיח את המשפט הבא:
	\begin{theorem}
		לכל
		\(p\in \R^n\),
		פרט לקבוצה ממידה 0, \(L_p\) היא פונקציית מורס.
	\end{theorem}


	\section{גאומטריה דיפרנציאלית}
	
	\begin{definition*}
		בכל נקודה
		$p\in M$
		יש הכלה של מרחבים משיקים
		\(\mrm{T}_p M \subset \mrm{T}_p \R^n\)
		(כדאי לחשוב כאן על ההגדרה באמצעות מהירויות של מסילות).
		ניתן להסתכל על המרחב הוקטורי המאונך אליו, ולקבל את \textbf{המרחב הנורמלי} בנקודה:
		\[
			\mrm{N}_p M
			= \left(\mrm{T}_p M\right)^\bot
			= \left\{
				v \in \mrm{T}_p \R^n
				\mid
				\forall u \in \mrm{T}_p M: \langle v,u \rangle=0
			\right\}
		\]
		את כל המרחבים הנורמליים ניתן לחבר לכדי \textbf{האגד הנורמלי}:
		\[
			\mrm{N} M
			= \coprod_{p \in M} \mrm{N}_p M
			\subset \mrm{T} \R^n
		\]
		
		א
	\end{definition*}
	
	

\end{document}